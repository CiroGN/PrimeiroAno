\documentclass{article}
\usepackage{pgfplots}
\pgfplotsset{width=7cm,compat=1.8}
\usepackage{pgfplotstable}
\renewcommand*{\familydefault}{\sfdefault}
\usepackage{sfmath}
    \usepackage{pgfplots}
\usetikzlibrary{positioning, backgrounds}
\usepackage{rotating}
\usepackage[utf8]{inputenc}
\usepackage[brazil]{babel}
\usepackage{adjustbox}
\usepackage{subcaption}
\usepackage{graphicx} % Required for inserting images
\usetikzlibrary{patterns}
\usepackage[letterpaper,top=2cm,bottom=2cm,left=3cm,right=3cm,marginparwidth=1.75cm]{geometry}

\title{Projeto Fila}
\author{CiroGuilhermeNass}
\date{November 2023}

\begin{document}

\maketitle
\begin{center}
ATENÇÃO: os graficos avaliam as médias de 10 em 10 execuções por questões de hardware.

\section{Desempenho por Template}
\end{center}
Usando o template de vector para a criação das filas, obtivemos o maior tempo de processamento das filas quando comparado aos outros métodos, após o 6° digito, é impresso o expoente "e+06" (que representa a casa dos milhões) ao criar as filas e "e+09" (que representa a casa dos bilhões) para retirar os elementos das filas (valores presentes em todos os processos sem exceção), como a diferença dos valores é altíssima, optei por não incluir no gráfico, \textbf{mas tome nota de que os valores da linha em vermelho são maiores quando comparados a linha azul}. todos os valores estão em microsegundos.
\begin{center}
\begin{tikzpicture}
\begin{axis}[
     title={\textbf{Medias das filas}},
     ylabel={Tempo de processamento (microsegundos)},
     xlabel={Processamentos},
     legend style={
                        at={(0.5,-0.3)},
                        anchor=north,
                        legend columns=1,
                        /tikz/every even column/.append style={column sep=0.5 cm}
                    },
     nodes near coords={
                        \pgfmathprintnumber[precision=1]{\pgfplotspointmeta}
                        }
     ]
	\addplot+[sharp plot] coordinates 
		{(1,3.62255) (2,1.89632) (3,3.5689) (4,3.3593) (5,3.73745) (6,3.5821) (7,3.47305) (8,4.69475) (9,3.32595) (10,3.7429)};
    \addplot+[sharp plot] coordinates 
		{(1,1.89632) (2,1.87939) (3,1.80937) (4,1.71906) (5,1.8125) (6,1.79981) (7,1.80175) (8,2.04842) (9,1.78442) (10,1.81431)};
    \legend{velocidade de criação da fila (milhões), velocidade na saida da fila (bilhões)}
\end{axis}
\end{tikzpicture}
\end{center}

\pagebreak
\begin{center}
\section{Medias por Queue}
\end{center}
Quando comparamos o método Queue aos outros, é evidente sua velocidade superior de processamento, milhão para representação dos valores de criação e milhar para a retirada dos elementos nas filas, levando mais tempo para montar as filas que retirar os elementos delas. Todos os valores estão em microsegundos.
\begin{center}
\begin{tikzpicture}
\begin{axis}[
     title={\textbf{Processamentos das filas}},
     ylabel={Tempo de processamento (microsegundos)},
     xlabel={Processamentos},
     legend style={
                        at={(0.5,-0.3)},
                        anchor=north,
                        legend columns=1,
                        /tikz/every even column/.append style={column sep=0.5 cm}
                    }, 
     nodes near coords={
                        \pgfmathprintnumber[precision=1]{\pgfplotspointmeta}
                        }
     ]
	\addplot+[sharp plot] coordinates 
		{(1,1151380) (2,1361840) (3,3762230) (4,1211800) (5,1368650) (6,1391330) (7,1447060) (8,1350920) (9,1203760) (10,2607000)};
    \addplot+[sharp plot] coordinates 
		{(1,753710) (2,595270) (3,1180360) (4,621880) (5,655780) (6,785890) (7,872540) (8,675700) (9,835160) (10,954960)};
    \legend{velocidade de criação da fila, velocidade na saida da fila}
\end{axis}
\end{tikzpicture}
\end{center}

\pagebreak
\begin{center}
\section{Medias por Filas Prioritarias}
\end{center}
É notado que no método por Filas de Prioridade, há uma semelhança com o tempo do método template de vector, adotando novamente expoentes após os seis primeiros digitos dos valores, como "e+06" para a criação de filas e, desta vez, diferentemente, "e+07" para retirada dos elementos das filas, e como novamente esta mudança de diferença torna os valores altos para representa-los em um grafico, aviso novamente que os valores da linha vermelha \textbf{são maiores} que os da linha azul. Todos os valores estão em microsegundos.
\begin{center}
\begin{tikzpicture}
\begin{axis}[
     title={\textbf{Processamentos das filas}},
     ylabel={Tempo de processamento (microsegundos)},
     xlabel={Processamentos},
     legend style={
                        at={(0.5,-0.3)},
                        anchor=north,
                        legend columns=1,
                        /tikz/every even column/.append style={column sep=0.5 cm}
                    }, 
     nodes near coords={
                        \pgfmathprintnumber[precision=1]{\pgfplotspointmeta}
                        }
     ]
	\addplot+[sharp plot] coordinates 
		{(1,6.99912) (2,7.20374) (3,9.85838) (4,6.8597) (5,7.18513) (6,7.17091) (7,7.18254) (8,7.19831) (9,7.15775) (10,7.16944)};
    \addplot+[sharp plot] coordinates 
		{(1,2.12775) (2,2.25872) (3,2.82095) (4,2.08541) (5,2.14652) (6,2.14963) (7,2.15986) (8,2.21717) (9,2.16128) (10,2.1865)};
    \legend{velocidade de criação da fila (milhar), velocidade na saida da fila (dez milhões)}
\end{axis}
\end{tikzpicture}
\section{Propriedades do Sistema}
\end{center}
Arquitetura: x86 64
Modo(s) operacional da CPU: 32-bit, 64-bit
Address sizes: 36 bits physical, 48 bits virtual
Ordem dos bytes: Little Endian
CPU(s): 2
Lista de CPU(s) on-line: 0,1
ID de fornecedor: GenuineIntel
Nome do modelo: Intel(R) Core(TM)2 Duo CPU     T5870  @2.00GHz
Família da CPU: 6
Modelo: 15
Thread(s) per núcleo: 1
Núcleo(s) por soquete: 2
Soquete(s): 1
Step: 13
Frequency boost: enabled
CPU(s) scaling MHz:40%
CPU MHz máx.: 2001,0000
CPU MHz mín.: 800,0000

\end{document}
